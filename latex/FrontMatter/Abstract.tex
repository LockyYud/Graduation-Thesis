\chapter*{Tóm tắt}
\addcontentsline{toc}{chapter}{Abstract}
Xương sông của Chatbot hiện nay - các mô hình ngôn ngữ lớn (Large Language Models - LLMs) - gần đây đã chứng minh khả năng hiểu ngôn ngữ đáng kể và hội thoại gần giống con người. Tuy nhiên, việc chỉ dựa vào LLMs cho các tác vụ hỏi-đáp là không đủ vì chúng bị giới hạn bởi dữ liệu huấn luyện và dễ tạo ra thông tin không chính xác, đặc biệt là trong các lĩnh vực cụ thể. Các kỹ thuật tạo nội dung tăng cường truy xuất (Retrieval-Augmented Generation - RAG) nhằm giải quyết những hạn chế đó bằng cách đưa vào các nguồn tri thức bên ngoài, cải thiện độ chính xác của câu trả lời do LLM tạo ra.

Bài khóa luận này phân tích sơ bộ điểm mạnh và điểm yếu của các hệ thống RAG sử dụng cơ sở dữ liệu vector so với cơ sở dữ liệu đồ thị. Kết quả cho thấy, các hệ thống RAG dựa trên vector vượt trội trong việc trả lời các câu hỏi chung và phù hợp hơn với các LLM có khả năng xử lý quy tắc phức tạp hạn chế. Ngược lại, các hệ thống dựa trên đồ thị hiệu quả hơn với các câu hỏi phức tạp đòi hỏi dữ liệu từ nhiều tài liệu. Từ đó đề xuất phương pháp kết hợp cả hai cơ sở dữ liệu để cải thiện hiệu suất của hệ thống RAG.

% Mục tiêu của khóa luận này