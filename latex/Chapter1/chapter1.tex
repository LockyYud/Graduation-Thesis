\chapter{Đặt vấn đề}
\label{chapter:introduction}
% -------------------------------------------------------------------
% BACKGROUND
% -------------------------------------------------------------------
\section{Bối cảnh và vấn đề}
\label{section:background}


Mặc dù việc tìm cách tạo ra một thứ có thể hiểu và giao tiếp với người tạo ra nó đã ăn sâu vào lịch sử loài người, Alan Turing được cho là người đầu tiên hình thành ý tưởng về chatbot vào năm 1950, khi ông đặt câu hỏi: “Máy móc có thể suy nghĩ không?”. Mô tả của Turing về hành vi của một cỗ máy thông minh gợi lên khái niệm chatbot mà chúng ta thường hiểu ngày nay \cite{Turing1950-TURCMA}.


Chatbot đã phát triển cùng với sự gia tăng dần khả năng tính toán và những tiến bộ trong các công cụ và kỹ thuật xử lý ngôn ngữ tự nhiên (NLP). Việc triển khai chatbot đầu tiên, dựa nhiều vào các quy tắc ngôn ngữ và kỹ thuật khớp mẫu, đã được thực hiện vào năm 1966 với sự ra đời của ELIZA . Chatbot này có thể giao tiếp với người dùng thông qua chương trình khớp từ khóa, tìm kiếm các quy tắc chuyển đổi thích hợp để tái cấu trúc đầu vào và đưa ra phản hồi, tức là câu trả lời cho người dùng. ELIZA là một hệ thống mang tính bước ngoặt, khuyến khích nghiên cứu sâu hơn trong lĩnh vực này. Tuy nhiên, phạm vi hiểu biết của ELIZA bị giới hạn vì nó phụ thuộc rất ít vào việc nhận diện ngữ cảnh và các quy tắc khớp mẫu thường không linh hoạt để triển khai trong các lĩnh vực mới \cite{weizenbaum1966eliza,shum2018elizaxiaoicechallengesopportunities,zemcik2019chatbots}.


Một bước tiến quan trọng trong sự phát triển của chatbot vào những năm 1980 là việc sử dụng trí tuệ nhân tạo. A.L.I.C.E. (Artificial Intelligent Internet Computer Entity) dựa trên ngôn ngữ đánh dấu trí tuệ nhân tạo (AIML), một phần mở rộng của XML. AIML được phát triển đặc biệt để cho phép thêm kiến thức về mẫu hội thoại vào phần mềm của A.L.I.C.E., giúp mở rộng cơ sở dữ liệu kiến thức. Các đối tượng dữ liệu trong AIML bao gồm các chủ đề  và danh mục. Danh mục là đơn vị kiến thức cơ bản, bao gồm các quy tắc để khớp đầu vào của người dùng với đầu ra của chatbot. Đầu vào của người dùng được biểu diễn dưới dạng mẫu quy tắc, trong khi đầu ra của chatbot được xác định bằng mẫu quy tắc trong cơ sở kiến thức của A.L.I.C.E. Việc bổ sung các đối tượng dữ liệu mới vào AIML đại diện cho một cải tiến đáng kể so với các hệ thống khớp mẫu trước đây vì cơ sở dữ liệu kiến thức dễ dàng mở rộng. Hơn nữa, ChatScript, kế thừa từ AIML, cũng là công nghệ nền tảng đằng sau các chatbot đoạt giải Loebner. Ý tưởng chính của công nghệ này là khớp các đầu vào văn bản từ người dùng với một chủ đề, và mỗi chủ đề sẽ có các quy tắc cụ thể để tạo ra phản hồi. ChatScript đã mở ra một kỷ nguyên mới trong sự phát triển công nghệ chatbot, bắt đầu chuyển trọng tâm sang phân tích ngữ nghĩa và hiểu biết \cite{cahn2017chatbot,bradesko2012survey, wilcox2014winning, abushawar2015alice, shum2018elizaxiaoicechallengesopportunities,zemcik2019chatbots}.


Hạn chế chính của việc dựa vào các quy tắc và khớp mẫu trong chatbot là chúng phụ thuộc vào lĩnh vực, khiến chúng trở nên kém linh hoạt vì phải dựa vào các quy tắc được viết thủ công cho các lĩnh vực cụ thể. Với những tiến bộ gần đây trong các kỹ thuật học máy và công cụ xử lý ngôn ngữ tự nhiên, kết hợp với khả năng tính toán mạnh mẽ, các khung công việc và thuật toán mới đã được tạo ra để triển khai các chatbot “nâng cao” mà không phụ thuộc vào quy tắc và kỹ thuật khớp mẫu, đồng thời khuyến khích việc sử dụng chatbot trong thương mại. Việc áp dụng các thuật toán học máy vào chatbot đã được nghiên cứu, và những kiến trúc chatbot mới đã xuất hiện.


Ứng dụng của chatbot đã mở rộng với sự xuất hiện của các thuật toán học sâu (Deep Learning). Một trong những ứng dụng mới và thú vị nhất là sự phát triển của các trợ lý cá nhân thông minh (như Alexa của Amazon, Siri của Apple, Google Assistant của Google, Cortana của Microsoft, và Watson của IBM). Các trợ lý cá nhân thông minh hoặc tác nhân hội thoại này thường có thể giao tiếp với người dùng thông qua giọng nói và thường được tích hợp trong điện thoại thông minh, đồng hồ thông minh, loa và màn hình gia đình chuyên dụng, thậm chí cả xe hơi. Ví dụ, khi người dùng nói một từ hoặc cụm từ đánh thức, thiết bị sẽ kích hoạt và trợ lý cá nhân thông minh bắt đầu lắng nghe. Thông qua việc hiểu ngôn ngữ tự nhiên, trợ lý có thể hiểu các lệnh và trả lời yêu cầu của người dùng, thường bằng cách cung cấp thông tin (ví dụ: “Alexa, thời tiết hôm nay ở Los Angeles thế nào?” – “Ở Los Angeles, trời nắng và nhiệt độ là 75°F”), hoặc thực hiện các nhiệm vụ (ví dụ: “Ok Google, phát danh sách nhạc buổi sáng của tôi trên Spotify”). Tuy nhiên, nhiệm vụ hiểu ngôn ngữ của con người vẫn là một thách thức lớn vì sự đa dạng về giọng điệu, vùng miền, địa phương, và thậm chí là cách nói cá nhân.


Tất cả các trợ lý cá nhân thông minh đều có các đặc điểm cốt lõi giống nhau về công nghệ sử dụng, giao diện người dùng và chức năng. Tuy nhiên, một số chatbot có tính cách phát triển hơn, và những chatbot phát triển nhất có thể cung cấp giải trí chứ không chỉ hỗ trợ các công việc hàng ngày; những chatbot này được gọi là chatbot xã hội. Một ví dụ thú vị về chatbot xã hội là XiaoIce của Microsoft. XiaoIce được thiết kế để trở thành một người bạn đồng hành lâu dài với người dùng, và để đạt được mức độ gắn kết cao, nó được xây dựng với tính cách, chỉ số thông minh (IQ) và chỉ số cảm xúc (EQ). Các khả năng IQ bao gồm mô hình hóa kiến thức và trí nhớ, hiểu hình ảnh và ngôn ngữ tự nhiên, lý luận, sáng tạo, và dự đoán. Đây là những thành phần quan trọng trong việc phát triển khả năng hội thoại, đáp ứng các nhu cầu cụ thể của người dùng và hỗ trợ họ. Khả năng quan trọng và phức tạp nhất là Core Chat, cho phép trò chuyện dài và trong các lĩnh vực mở với người dùng. Đồng cảm và kỹ năng xã hội là hai thành phần quan trọng của EQ. Công cụ hội thoại của XiaoIce sử dụng một trình quản lý hội thoại để theo dõi trạng thái của cuộc hội thoại và lựa chọn giữa Core Chat (thành phần tạo hội thoại mở) hoặc kỹ năng hội thoại để tạo phản hồi. Do đó, mô hình tích hợp cả khả năng truy xuất thông tin và khả năng tạo hội thoại \cite{dormehl2018xiaoi, spencer2018xiaoi, zhou2019xiaoi}.


Trong bối cảnh các mô hình ngôn ngữ lớn - \gls{llm} ngày càng phát triển, việc áp dụng chúng vào trong các Chatbot trở thành các trợ lý, công cụ hỗ trợ càng trở nên quan trọng và trở thành một trong các lĩnh vực quan trọng để nghiên cứu. Mặc dù khả năng của các Chatbot sử dụng LLMs như GPT, Gemini, Claude chứng tỏ được năng lực mạnh mẽ trong nhiệm vụ hỏi đáp nhưng vẫn còn một số vấn đề nổi cộm còn tồn đọng khi vẫn phải đối mặt với những hạn chế cố hữu, chẳng hạn như ảo giác và các kiến thức được học đã lỗi thời cũng như chưa tập trung vào trong các miền tri thức cụ thể. Với khả năng mạnh mẽ  trong việc cung cấp thông tin bổ trợ mới nhất và hữu ích, các Chatbot sử dụng mô hình ngôn ngữ lớn tăng cường truy xuất (\gls{rag}) đã xuất hiện để khai thác các cơ sở kiến thức bên ngoài và có thẩm quyền, thay vì chỉ dựa vào kiến thức bên trong của mô hình, nhằm tăng cường chất lượng phản hồi của \gls{llm}.


Mặc dù vậy \gls{rag} vẫn còn có một số hạn chế như phụ thuộc vào các mô hình nhúng văn bản (Embedding models), độ chính xác của việc trích xuất thông tin, và khả năng trả lời các câu hỏi cần suy luận. Để giải quyết một phần thách thực được đề cập, trong khóa luận này tôi đề xuất một phương pháp cải tiến \gls{rag} bằng cách kết hợp hiệu quả kiến thức phi cấu trúc từ văn bản và các thông tin chi tiết có cấu trúc từ đồ thị tri thức để nâng cao chất lượng phản hồi của Chatbot cho nhiệm vụ hỏi đáp trong một lĩnh vực cụ thể với các tài liệu đã được chuẩn hóa.


\section{Mục tiêu và phạm vi nghiên cứu}
\label{section:objective_and_scope}
Mục tiêu chính của khóa luận này là nâng cao khả năng trả lời câu hỏi của Chatbot trong các miền tri thức với tài liệu đã được chuẩn hóa thông qua việc phát triển và cải thiện hệ thống \gls{rag} nhằm cải thiện đáng kể độ chính xác và độ tin cậy của các phản hồi được tạo ra bởi các mô hình ngôn ngữ lớn (LLMs) - xương sống của Chatbot hiện nay - trong nhiệm vụ Q\&A. Cụ thể khóa luận sẽ tập trung vào việc giải quyết các hạn chế của các mô hình ngôn ngữ lớn bao gồm: giảm sự xuất hiện của hiện tượng "hallucination" (khi mô hình đưa ra thông tin không chính xác hoặc không tồn tại), xử lý hiệu quả các truy vấn phức tạp và đòi hỏi suy luận, đảm bảo các biện pháp bảo mật mạnh mẽ để bảo vệ thông tin nhạy cảm. Phạm vi khóa luận bao gồm việc nghiên cứu và triển khai một phương pháp, hệ thống \gls{rag} trong phạm vi một miền tri thức cụ thể. Nghiên cứu đảm bảo rằng hệ thống có thể áp dụng hiệu quả trong các miền tri thức thực tế, chẳng hạn như các miền tri thức về giáo dục bao gồm Lịch Sử, Địa Lý, Sinh Học, \dots với các tài liệu là sách giáo khoa đã được chuẩn hóa. (sửa lại phạm vi)


\subsection{Mục tiêu}
\label{subsection:objective}
Khóa luận này đặt mục tiêu chính là phát triển và triển khai một hệ thống Retrieval-Augmented Generation (RAG) tích hợp cả đồ thị tri thức và văn bản, nhằm khắc phục những hạn chế của các mô hình ngôn ngữ lớn hiện tại cũng như các phương pháp \gls{rag} trước đây trong việc giải quyết các câu hỏi thuộc một miền tri thức cụ thể với các tài liệu đã được chuẩn hóa. Các hệ thống \gls{rag} truyền thống sử dụng cơ sở dữ liệu vector thường mang lại thông tin rộng rãi, phù hợp với những câu hỏi tổng quát. Ngược lại, \gls{rag} dựa trên cơ sở dữ liệu đồ thị lại cung cấp cấu trúc phong phú và logic, giúp nâng cao khả năng xử lý các câu hỏi phức tạp, yêu cầu nhiều quy tắc. Sự kết hợp giữa hai phương pháp này sẽ tạo ra một hệ thống cải thiện đáng kể độ chính xác trong câu trả lời của Chatbot. Cụ thể, các mục tiêu bao gồm:


\begin{enumerate}
    \item \textbf{Nâng cao năng lực của mô hình ngôn ngữ:} Tích hợp các cơ chế truy xuất nâng cao để giảm thiểu tình trạng ảo giác thông tin, đảm bảo rằng mô hình ngôn ngữ tạo ra các phản hồi chính xác và đáng tin cậy về mặt dữ liệu.


    \item \textbf{Cải thiện khả năng xử lý các truy vấn phức tạp:} Phát triển các khả năng xử lý dữ liệu khó (long-tail data) và các truy vấn đa bước, cho phép hệ thống suy luận dựa trên nhiều nguồn bằng chứng hỗ trợ cho các câu hỏi phức tạp trong doanh nghiệp.
    \item \textbf{Giải quyết các vấn đề về quyền riêng tư:} Triển khai các quy trình xử lý và truy xuất dữ liệu mạnh mẽ để giảm thiểu rủi ro rò rỉ dữ liệu, đảm bảo tính bảo mật cho các tập dữ liệu độc quyền và nhạy cảm.
    \item \textbf{Hỗ trợ các ứng dụng thực tiễn trong doanh nghiệp:} Thiết kế hệ thống linh hoạt và có khả năng ứng dụng trong nhiều nhiệm vụ doanh nghiệp, bao gồm trả lời câu hỏi dựa trên tài liệu, truy xuất thông tin và tạo nội dung.
\end{enumerate}


Nghiên cứu này nhằm nâng cao độ chính xác phản hồi của Chatbot khi được triển khai với các tài liệu đã được chuẩn hóa thuộc một miền tri thức cụ thể. Bằng cách đạt được các mục tiêu này, khóa luận sẽ đóng góp vào.
\subsection{Phạm vi nghiên cứu}
\label{subsection:scope}
Phạm vi của khóa luận này tập trung vào việc nghiên cứu sẽ tập trung vào cải thiện chất lượng phản hồi của Chatbot trong nhiệm vụ hỏi đáp trong một miền tri thức cụ thể với các tài liệu đã được chuẩn hóa. Vậy nên việc phát triển hệ thống RAG được đề xuất là chủ yếu và giao diện của người dùng là một phần không thể thiếu của Chatbot, nhưng không phải là mục tiêu chính của khóa luận này nên sẽ được xây dựng cơ bản đủ để thực hiện các thử nghiệm và đánh giá. Thêm vào đó, sẽ không tập trung vào việc cải thiện các khía cạnh khác của Chatbot như xử lý lịch sử hội thoại, tương tác giọng nói, hoặc các ứng dụng khác của Chatbot. Thông qua các nghiên cứu chi tiết này, khóa luận hướng tới việc đóng góp vào sự phát triển của các hệ thống \gls{rag} trong các ứng dụng hỏi đáp trong các miền tri thức cụ thể, đặc biệt là trong lĩnh vực giáo dục.


\begin{enumerate}
    \item Thiết kế và kiến trúc hệ thống: Khám phá chi tiết và tài liệu hóa kiến trúc hệ thống của hệ thống \gls{rag} được đề xuất, bao gồm quy trình trích xuất dữ liệu, cập nhật dữ liệu, và các cơ chế truy vấn cho cả dữ liệu có cấu trúc và không cấu trúc.
    \item Triển khai cơ chế truy xuất thông tin: Phát triển và tích hợp các kỹ thuật truy xuất thông tin tiên tiến nhằm nâng cao độ chính xác và độ tin cậy của các phản hồi được tạo bởi mô hình ngôn ngữ. Điều này bao gồm việc thiết kế các phương pháp lập chỉ mục, thuật toán tìm kiếm hiệu quả, và quy trình bổ sung dữ liệu một cách liền mạch.
    \item Thiết kế giao diện và trải nghiệm người dùng: Thiết kế một giao diện thân thiện và trực quan để người dùng tương tác với hệ thống. Điều này bao gồm việc tạo các cơ chế nhập liệu rõ ràng cho truy vấn, đảm bảo đầu ra dễ hiểu, và nâng cao trải nghiệm tổng thể của người dùng.
\end{enumerate}
\section{Cấu trúc của khóa luận}
Chương mở đầu này đã cung cấp một sự hiểu biết cơ bản về vấn đề nghiên cứu và tầm quan trọng của nó. Các chương tiếp theo sẽ đi sâu vào các khía cạnh khác nhau của khóa luận:
\begin{itemize}
    \item \textbf{Chương 2} cung cấp một cái nhìn tổng quan về các nghiên cứu trước đây liên quan đến Chatbot, hệ thống Retrieval-Augmented Generation và sự tích hợp của chúng với các mô hình ngôn ngữ lớn. Nội dung nhấn mạnh các tiến bộ quan trọng, những thách thức hiện tại, và khoảng trống trong tài liệu nghiên cứu mục tiêu của khóa luận này muốn giải quyết.
    \item \textbf{Chương 3} trình bày thiết kế và triển khai hệ thống \gls{rag} kết hợp với đồ thị tri thức được đề xuất. Nội dung bao gồm kiến trúc tổng thể của hệ thống, từ các quy trình truy xuất và xử lý dữ liệu, cơ chế xử lý truy vấn, đến các chiến lược tạo phản hồi. Chương cũng đề cập đến các khía cạnh kỹ thuật như lưu trữ siêu dữ liệu, chia nhỏ dữ liệu (data chunking), xây dựng cơ sở dữ liệu đồ thị (construction graph database), và lưu trữ vector nhúng (embedding vector).
    \item \textbf{Chương 4} cung cấp một đánh giá chi tiết về hệ thống \gls{rag} được đề xuất thông qua các thử nghiệm và kiểm tra mở rộng. Nội dung bao gồm thiết lập thí nghiệm, như môi trường phần cứng và phần mềm, bộ dữ liệu và các loại truy vấn được sử dụng. Chương phân tích hiệu suất của hệ thống về độ chính xác, độ tin cậy và khả năng xử lý các truy vấn có độ phức tạp khác nhau trên bộ tài liệu thuộc miền tri thức cụ thể. Kết quả làm nổi bật các điểm mạnh của hệ thống trong việc cung cấp thông tin chính xác và liên quan, đồng thời chỉ ra các lĩnh vực cần cải thiện. Đánh giá này chứng minh tính ứng dụng thực tế của hệ thống trong các kịch bản kinh doanh và định hướng cho các nỗ lực phát triển trong tương lai.
    \item \textbf{Chương 5} tóm tắt các phát hiện chính của nghiên cứu và phân tích những điểm mạnh cũng như hạn chế của hệ thống được đề xuất. Đồng thời, nội dung cũng đưa ra các tác động tiềm năng của hệ thống trong các môi trường kinh doanh thực tế và xác định các lĩnh vực cần cải tiến và nghiên cứu trong tương lai.
\end{itemize}