\chapter{Kết luận}
\label{chap:conclusions}
\section{Kết luận}
\label{sec:conclusion}
Trong bối cảnh mà mô hình ngôn ngữ lớn càng ngày càng chứng tỏ được tiềm năng lớn, việc ứng dụng vào các ứng dụng vào các bài toán thực tế trở thành một chủ đề được chú trọng phát triển và nghiên cứu, đặc biệt là trong lĩnh vực hỏi đáp khi mà các Chatbot hỗ trợ như GPT, Gemini,\dots ngày càng trở nên phổ biến. Việc đảm bảo chất lượng phản hồi của Chatbot trở thành yếu tố cốt lõi để nâng cao hiệu quả và tính ứng dụng của chúng. Khóa luận này nghiên cứu về phương pháp cải thiện chất lượng phản hồi của Chatbot thông qua việc sử dụng phương pháp tạo sinh tăng cường truy xuất kết hợp với đồ thị tri thức.

Cụ thể, khóa luận đã nghiên cứu và đề xuất một phương pháp tích hợp dựa trên mô hình tăng cường truy xuất tạo sinh (Retrieval-Augmented Generation) với đồ thị tri thức (Knowledge Graph) nhằm nâng cao hiệu quả phản hồi của Chatbot trong các miền tri thức cụ thể với các tài liệu đã được chuẩn hóa thuộc miền tri thức đó. Phương pháp được đề xuất gồm có các phần xây dựng cơ sở dữ liệu đồ thị, và tạo sinh phản hồi từ thông tin truy xuất thông qua suy luận trên đồ thị tri thức.

Để kiểm chứng phương pháp đề xuất, tôi đã phát triển hệ thống thực nghiệm và thực nghiệm với dữ liệu là các đề thi THPT Lịch sử các năm. Kết quả từ các thử nghiệm đã cho thấy rằng hệ thống RAG được đề xuất đã khắc phục một số hạn chế hiện tại của các mô hình ngôn ngữ lớn và phương pháp RAG truyền thống, như hiện tượng ảo giác thông tin, độ chính xác trong xử lý truy vấn đa bước. Kết quả đạt được không chỉ nâng cao chất lượng phản hồi của Chatbot mà còn mở ra các ứng dụng tiềm năng trong các lĩnh vực thực tế như giáo dục, luật, và tài chính.

Tổng quan, khóa luận đã đạt được các mục tiêu đề ra, đồng thời đóng góp thêm vào lĩnh vực nghiên cứu về Chatbot và các ứng dụng của mô hình ngôn ngữ lớn trong thực tế.
\section{Định hướng phát triển trong tương lai}
\label{sec:futurework}
Mặc dù hệ thống được đề xuất trong khóa luận đã cho thấy những kết quả đầy hứa hẹn, tuy nhiên vẫn còn một số điểm cần cải thiện và mở rộng trong tương lai để tăng cường hơn nữa các khả năng của nó:
\begin{enumerate}
    \item \textbf{Mở rộng phạm vi tài liệu}: Hiện tại, hệ thống chỉ sử dụng dữ liệu từ các tài liệu được chuẩn hóa với nhau, cụ thể như là sách giáo khoa. Trong tương lai, việc mở rộng phạm vi tài liệu từ nhiều nguồn khác nhau như báo chí, sách báo, tạp chí,\dots sẽ giúp hệ thống trở nên phong phú hơn và đa dạng hơn.
    \item \textbf{Mở rộng cơ sở dữ liệu đồ thị}: Cơ sở dữ liệu đồ thị mới chỉ được xây dựng từ các tài liệu nên vẫn còn hạn chế về số lượng thông tin, tri thức. Việc tích hợp với các nền tảng cơ sở dữ liệu đồ thị tri thức với lượng thông tin phong phú và cập nhật liên tục như Wikidata, DBpedia,\dots  sẽ giúp hiệu quả truy xuất, suy luận được nâng cao.
    \item \textbf{Mở rộng kiến trúc hệ thống}: Trong khóa luận, hệ thống chỉ tập trung vào việc cải thiện khả năng trả lời câu hỏi. Trong tương lai, việc mở rộng kiến trúc hệ thống để hỗ trợ nhiều mô hình ngôn ngữ lớn khác nhau, cũng như cung cấp các dịch vụ khác như tìm kiếm, gợi ý,\dots sẽ giúp hệ thống trở nên linh hoạt và đa dạng hơn.
\end{enumerate}

